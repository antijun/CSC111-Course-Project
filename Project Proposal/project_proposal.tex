\documentclass[fontsize=11pt]{article}
\usepackage{amsmath}
\usepackage[utf8]{inputenc}
\usepackage[margin=0.75in]{geometry}

\title{CSC111 Project Proposal: Music Genre Proximity Tree}
\author{David Wu, Kevin Hu}
\date{Tuesday, March 16, 2021}

\begin{document}
\maketitle

\section*{Problem Description and Research Question}

To many of us, music is a large part of our lives, helping us get through the day. It comes in many styles or genres, such as rock, pop, hip-hop, etc, with each genre appealing to different people. Songs can even have multiple genres, with matching characteristics of different styles. Songs can also be given second descriptors or adjectives, such as warm, aggressive, and depressing, which is another way to categorize songs. However, with the vast amount of music styles and songs, it may be hard for someone new to find the genre they like, or for someone to discover new genres they may enjoy. Personally, we find it difficult and time-consuming to explore individual songs and artists on apps such as Spotify or Apple Music, and to create our own playlists from our collected songs. \textbf{Our goal is to provide users with the best possible song recommendations, given the current preferences of the user. This may be in the form of a genre the user likes, or by using the descriptors of an album provided by the user.} We would then provide a tree that would order albums based on how well they matched to the user's input. The albums that matched the best would be placed at the top, with each branch of the tree allowing the user to explore a specific musical pathway, based on if they liked earlier albums in the tree. With this visualization, users can more easily find and potentially discover new songs and albums they can enjoy, without searching through individual songs on music streaming sites.

\section*{Computational Plan}

In terms of computations, we'll be filtering our data set by genres and by album descriptors. We'll first create a simple tree that branches off by genre(i.e. rock can branch off into classic rock, punk rock, etc), which the user can explore on their own, or after they get a recommendation. This tree can be created by sorting by keywords(i.e. finding the keyword 'rock' for one subtree) and adding subtrees by these keywords. The other tree we will be creating is our recommendation tree, which will have two different settings. The user can decide to input a genre they like or an album they like. If the user first inputs an album, the descriptors of that album will be taken and compared to every other album. Albums with the most matches would then be added as subtrees of the recommendation tree. Then, with each recommended album, the descriptors of those albums will be taken and compared, to then create another line of recommendation. To make sure the same albums aren't recommended over and over, albums already existing in the upper levels of the tree will not be in the comparison at lower levels of the tree. If the user first inputs a genre, the top albums of that genre will first be displayed(a top album being the most popular album, measured by the number of interactions with that album in our data set). Within those recommended albums, the descriptors of those albums will be taken, and the recommendation algorithm will then switch to matching the descriptors with other albums not already in the tree.
    To display our program, we will first use the tkinter library to display the genre tree, which the user can interact with to explore the different genres. We will then use the plotly library to display the recommendation trees on a separate window. To reduce the potential size of one tree, the recommendation tree will be restricted to a certain depth. If the user wants to explore a certain subtree more, they can click that subtree, which will then generate a new tree with the selected album as the new root. Previous recommendation trees will also be saved, so the user can have to option to go back if needed. 

\section*{References}

TODO

% NOTE: LaTeX does have a built-in way of generating references automatically,
% but it's a bit tricky to use so we STRONGLY recommend writing your references
% manually, using a standard academic format like APA or MLA.
% (E.g., https://owl.purdue.edu/owl/research_and_citation/apa_style/apa_formatting_and_style_guide/general_format.html)

\end{document}
